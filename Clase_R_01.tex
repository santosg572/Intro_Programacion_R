\documentclass{beamer}
%
% https://www.overleaf.com/13453549shwzhbhdrwmz#/51918539/
% Choose how your presentation looks.
%
% For more themes, color themes and font themes, see:
% http://deic.uab.es/~iblanes/beamer_gallery/index_by_theme.html
%

\usepackage[spanish]{babel}
\usepackage[utf8]{inputenc}

\mode<presentation>
{
  \usetheme{Frankfurt}      % or try Darmstadt, Madrid, Warsaw, ...
  \usecolortheme{crane} % or try albatross, beaver, crane, ...
  \usefonttheme{default}  % or try serif, structurebold, ...
%  \setbeamertemplate{navigation symbols}{}
%  \setbeamertemplate{caption}[numbered]
} 

%\usepackage[english]{babel}
%\usepackage[utf8x]{inputenc}

\title[Sistemas-Numeraci\'on]{Sistemas de Numeraci\'on}
\author{Leopoldo Gonz\'alez Santos}
\institute{Instituto de Neurobiolog\'ia \\ UNAM}
\date{Date of Presentation}

% https://es.wikipedia.org/wiki/N%C3%BAmero_natural

\begin{document}

\begin{frame}
  \titlepage
\end{frame}

% Uncomment these lines for an automatically generated outline.
%\begin{frame}{Outline}
%  \tableofcontents
%\end{frame}

\section{Introducci\'on}

\begin{frame}{Introduci\'on}

%\begin{itemize}
%  \item Your introduction goes here!
%  \item Use \texttt{itemize} to organize your main points.
%\end{itemize}

\begin{itemize}
\item Por definici\'on se dir\'a que cualquier miembro del siguiente conjunto, $\aleph = \{1, 2, 3, 4, … \}$, es un \textbf{número natural}. 

\vskip .5cm

Algunas características de los números naturales son:

\begin{enumerate}
\item Todo número mayor que 1 va después de otro número natural.
\item Entre dos números naturales siempre hay un número finito de naturales (interpretación de conjunto no denso).
\item Dado un n\'umero natural cualquiera, siempre existe otro natural mayor que este (interpretaci\'on de conjunto infinito).
\item Entre el n\'umero natural \textbf{a} y su sucesor \textbf{a+1}  no existe ning\'un n\'umero natural.  
\end{enumerate}

\end{itemize}
\vskip 1cm

%\begin{block}{Examples}
%Some examples of commonly used commands and features are included, to help %you get started.
%\end{block}

\end{frame}

\begin{frame}

\begin{itemize}
\item Por definici\'on se dir\'a que cualquier miembro del siguiente conjunto, $Z = \{... , -4, -3, -2, -1, 0, 1, 2, 3, 4, … \}$, es un \textbf{número entero}. 

\item Por definici\'on se dir\'a que cualquier miembro del siguiente conjunto, $Q = \{ \frac{p}{q} \hspace{2mm} | \hspace{2mm} p, q \in Z, \text{con} \hspace{2mm} q \neq 0 \}$, es un \textbf{número racional}. 

Ejemplos: 1/2, 4, 100/10000, .5, -.2

\item Por definici\'on se dir\'a que un n\'umero es irracional si noes racinal, es decir si el n\'umero no se puedde escribir como cociente entre dos n\umeros, i se representa como I. 

Ejemplos: 1/2, 4, 100/10000, .5, -.2



\end{itemize}

\end{frame}


\section{Some \LaTeX{} Examples}

\subsection{Tables and Figures}

\begin{frame}{Tables and Figures}

\begin{itemize}
\item Use \texttt{tabular} for basic tables --- see Table~\ref{tab:widgets}, for example.
\item You can upload a figure (JPEG, PNG or PDF) using the files menu. 
\item To include it in your document, use the \texttt{includegraphics} command (see the comment below in the source code).
\end{itemize}

% Commands to include a figure:
%\begin{figure}
%\includegraphics[width=\textwidth]{your-figure's-file-name}
%\caption{\label{fig:your-figure}Caption goes here.}
%\end{figure}

\begin{table}
\centering
\begin{tabular}{l|r}
Item & Quantity \\\hline
Widgets & 42 \\
Gadgets & 13
\end{tabular}
\caption{\label{tab:widgets}An example table.}
\end{table}

\end{frame}

\subsection{Mathematics}

\begin{frame}{Readable Mathematics}

Let $X_1, X_2, \ldots, X_n$ be a sequence of independent and identically distributed random variables with $\text{E}[X_i] = \mu$ and $\text{Var}[X_i] = \sigma^2 < \infty$, and let
$$S_n = \frac{X_1 + X_2 + \cdots + X_n}{n}
      = \frac{1}{n}\sum_{i}^{n} X_i$$
denote their mean. Then as $n$ approaches infinity, the random variables $\sqrt{n}(S_n - \mu)$ converge in distribution to a normal $\mathcal{N}(0, \sigma^2)$.

\end{frame}

\end{document}
