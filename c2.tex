\documentclass{beamer}

% https://matplotlib.org/users/pyplot\_tutorial.html#controlling-line-properties

\usetheme{Boadilla}
\usepackage{lmodern}
\usepackage{amsmath}
\usepackage{tikz}
\usepackage{amssymb}
\usetikzlibrary{trees}

% Instituto de Neurobiolog\'ia \ UNAM - M\'exico

\newcommand\nombre{ }
\newcommand\depto{Introducci\'on}
\newcommand\instituto{}
\newcommand\correo{}
\newcommand\iniciales{LGS}

\newtheorem{mytheorem}{Theorem}
\newtheorem{mylemma}{Lemma}
\newtheorem{mycorollary}{Corollary}
\theoremstyle{definition}
\newtheorem{mydefinition}{Definition}
\newtheorem{historylabel}{Historycal label}
\newtheorem{myexample}{Example}
\theoremstyle{remark}
\newtheorem{remark}{Remark}

\newcommand{\bR}{{\mathbf{R}}}
\newcommand{\bC}{{\mathbf{C}}}
\newcommand{\bT}{{\mathbf{T}}}
\newcommand{\bZ}{{\mathbf{Z}}}
\newcommand{\myemph}[1]{\alert{#1}}
\newcommand{\essinf}{\mathop{\mathrm{ess\,inf}}}
\renewcommand{\phi}{\varphi}
\newcommand{\eps}{\varepsilon}
\DeclareMathOperator{\adj}{adj}
\DeclareMathOperator{\Range}{Im}
\DeclareMathOperator{\mes}{mes}

\tikzstyle{thinarrow}=[densely dashed,very thin]

\title[Introducci\'on]
{\mbox{Introducci\'on al Lenguaje de Programaci\'on}\\mbox{PYTHON}}
\author[santosg572@gmail.com]{\nombre \
{\small \correo}}
\institute[\iniciales]{\instituto}
\date{\today}

\hypersetup{pdfkeywords={Hermitian Toeplitz matrices,eigenvectors,asymptotics}}

\AtBeginSection[] {
\begin{frame}{Contenido}
\tableofcontents[currentsection]
\end{frame}
}

%===============================================

\begin{document}

%\begin{frame}[label=titlepage]
%\titlepage

%\end{frame}
%===============================================

%\begin{frame}{Contents}
%\tableofcontents

%\end{frame}

%===============================================
% http://docs.python.org.ar/tutorial/2/introduction.html

%\section{Palabras reservadas en Python}

\begin{frame}{C2}

\textbf{$>>$ N\'umeros con exponentes}

\hfill

$>$ 1.2e3

\hfill

$>$ 1.2e-2

\hfill

$>$ 3.9+4.5i 

\hfill
\end{frame}

\begin{frame}{C2}

\textbf{$>>$ M\'odulo y Cociente Entero}

\hfill

$>$ 119 \%/\% 13

\hfill

$>$ 119 \%\% 13

\hfill

$>$ 9 \%\% 2

\hfill

$>$ 8 \%\% 2

\hfill

$>$ 15421 \%\% 7 == 0

   
\end{frame}

\begin{frame}{C2}

\textbf{$>>$ Redondeo}

\hfill

$>>$ floor(5.7)

\hfill

$>>$ ceiling(5.7)

\hfill

$>>$ rounded<-function(x) floor(x+0.5)

\hfill

$>>$ rounded(5.7)

\hfill

$>>$ rounded(5.4)

\end{frame}


\begin{frame}{C2}

\textbf{$>>$ Infinito y cosas que no son n\'umeros (NaN)}

\hfill

Inf

\hfill

-Inf:

\hfill

3/0

\hfill

-12/0

\hfill

exp(-Inf)

\hfill

0/Inf

\hfill

$(0:3)^{Inf}$

\hfill

NaN

\hfill





\end{frame}

\begin{frame}{C2}

0/0

\hfill

Inf-Inf

\hfill

Inf/Inf

\hfill

NA 

\hfill

is.finite(10)

\hfill

is.infinite(10)

\hfill

is.infinite(Inf)


\end{frame}

\begin{frame}{C2}

\textbf{$>>$ Missing values NA}

\hfill

x<-c(1:8,NA)

mean(x)

\hfill

mean(x,na.rm=T)

\hfill

is.na(x) 

\hfill

vmv<-c(1:6,NA,NA,9:12)

vmv

\hfill

seq(along=vmv)[is.na(vmv)]

\hfill

which(is.na(vmv))


\end{frame}

\begin{frame}{C2}

\hfill

vmv[is.na(vmv)] $<-$ 0

vmv

\hfill


vmv $<-$ c(1:6,NA,NA,9:12)

ifelse(is.na(vmv),0,vmv)

\hfill


\end{frame}


\begin{frame}{C2}

\textbf{$>>$ Operadores}

\hfill

+ \hspace{1mm}   - \hspace{1mm}  * \hspace{1mm}  /  \hspace{1mm}  \%\%  \hspace{1mm} (exponente)   \hspace{4mm} aritm\'eticos

\hfill

$> \hspace{3mm}  >= \hspace{3mm}   < \hspace{3mm}  <=  \hspace{3mm}  ==  \hspace{3mm}  != $ \hspace{4mm}	relacionales

\hfill

$!  \hspace{3mm}  \& \hspace{3mm}   |$	\hspace{4mm} 		logicos

\hfill

~  	 \hspace{4mm} 	modelo de formula

\hfill

$<- \hspace{3mm} ->$    asignamiento

\hfill

\$		indexamiento de listas

\hfill

: \hspace{4mm}	 crea secuencias



\end{frame}


\begin{frame}{C2}

\textbf{$>>$ Creando Vectores}

\hfill

y $<-$ 4.3

\hfill

z $<-$ y[-1]

\hfill

length(z)

\hfill

y $<-$ 10:16

\hfill

y $<-$ c(10, 11, 12, 13, 14, 15, 16)

\hfill

\end{frame}


\begin{frame}{C2}

y $<-$ scan()

1: 10

2: 11

3: 12

4: 13

5: 14

6: 15

7: 16

8:

Read 7 items

\hfill




\end{frame}


\begin{frame}{C2}

A $<-$ 1:10

\hfill

B $<- $c(2,4,8)

\hfill

A*B

\hfill

[1] 2 8 24 8 20 48 14 32 72 20

\end{frame}


\begin{frame}{C2}

\textbf{$>>$ Nombrando los elementos del Vector}


\hfill

counts $<-$ c(25,12,7,4,6,2,1,0,2))

[1] 25 12 7 4 6 2 1 0 2

\hfill

names(counts) $<-$ 0:8

\hfill

counts
0 1 2 3 4 5 6 7 8
25 12 7 4 6 2 1 0 2

\hfill

(st $<-$ table(rpois(2000,2.3)))

0 1 2 3 4 5 6 7 8 9

205 455 510 431 233 102 43 13 7 1

\hfill

as.vector(st)

[1] 205 455 510 431 233 102 43 13 7 1



\end{frame}


\begin{frame}{C2}

\textbf{$>>$ Funciones Vectoriales}


\hfill

\textbf{max(x)} -	maximum value in x

\textbf{min(x)} -	minimum value in x

\textbf{sum(x)} -	total of all the values in x

\textbf{mean(x)} -	arithmetic average of the values in x

\textbf{median(x)} - 	median value in x

\textbf{range(x)} -	vector of min x and max x

\textbf{var(x)}	-	sample variance of x

\textbf{cor(x,y)} -	correlation between vectors x and y

\textbf{sort(x)}	- a sorted version of x

\textbf{rank(x)} -	vector of the ranks of the values in x

\textbf{order(x)} -	an integer vector containing the permutation to sort x into ascending order

\textbf{quantile(x)} - 	vector containing the minimum, lower quartile, median, upper quartile, and
		maximum of x




\end{frame}


\begin{frame}{C2}

\textbf{cumsum(x)} -	vector containing the sum of all of the elements up to that point

\textbf{cumprod(x)} -	vector containing the product of all of the elements up to that point

\textbf{cummax(x)} -	vector of non-decreasing numbers which are the cumulative maxima of 
the values in x up to that point

\textbf{cummin(x)}	 - vector of non-increasing numbers which are the cumulative minima of the
		values in x up to that point
		
\textbf{colMeans(x)} -	column means of dataframe or matrix x

\textbf{colSums(x)} -	column totals of dataframe or matrix x

\textbf{rowMeans(x)} -	row means of dataframe or matrix x

\textbf{rowSums(x)} -	row totals of dataframe or matrix x


\end{frame}


\begin{frame}{C2}

\textbf{$>>$ Trabajando con vectores y Sub\'indices Logicos}



\hfill

x $<-$ 0:10

sum(x)

[1] 55

\hfill

sum($x<5$)

[1] 5

\hfill

sum(x[x<5])

[1] 10

\hfill

$x<5$

[1] TRUE TRUE TRUE TRUE TRUE FALSE FALSE FALSE FALSE

[10] FALSE FALSE

\hfill

1*($x<5$)

[1] 1 1 1 1 1 0 0 0 0 0 0

\hfill




\end{frame}


\begin{frame}{C2}

x*($x<5$)

[1] 0 1 2 3 4 0 0 0 0 0 0

\hfill

sum(x*(x $<$ 5))

\hfill

y<-c(8,3,5,7,6,6,8,9,2,3,9,4,10,4,11)

\hfill

sort(y)

[1] 2 3 3 4 4 5 6 6 7 8 8 9 9 10 11

\hfill

rev(sort(y))

[1] 11 10 9 9 8 8 7 6 6 5 4 4 3 3 2

\hfill

rev(sort(y))[2]

[1] 10

\hfill

rev(sort(y))[1:3]

[1] 11 10 9

\hfill

sum(rev(sort(y))[1:3])


\end{frame}


\begin{frame}{C2}

\textbf{$>>$ Direccionamiento con Vectores}

\hfill

y

[1] 8 3 5 7 6 6 8 9 2 3 9 4 10 4 11

\hfill

which($y>5$)

[1] 1 4 5 6 7 8 11 13 15

\hfill

y[$y>5$]

[1] 8 7 6 6 8 9 9 10 11

\hfill

length(y)

[1] 15

\hfill

length(y[$y>5$])

[1] 9

\hfill

xv $<-$ rnorm(1000,100,10)

\hfill

xv[seq(25,length(xv),25)]


\end{frame}


\begin{frame}{C2}

\textbf{$>>$ Encontrando los valores m\'as cercanos}

\hfill

which(abs(xv-108)==min(abs(xv-108)))

[1] 332

\hfill

xv[332]

[1] 108.0076

\hfill

closest $<-$ function(xv,sv) \{

xv[which(abs(xv-sv)==min(abs(xv-sv)))] \}

\hfill

closest(xv,108)

[1] 108.0076



\end{frame}


\begin{frame}{C2}

$>>$ Aritm\'etica Logica

\hfill

x $<-$ 0:6

\hfill

x $<$ 4

[1] TRUE TRUE TRUE TRUE FALSE FALSE FALSE

\hfill

all($ x>0$)

[1] FALSE

\hfill

any($ x<0$)

[1] FALSE

\hfill

sum($x<4$)

[1] 4



\end{frame}


\begin{frame}{C2}


$>>$ Repeticiones

\hfill

rep(9,5)

[1] 9 9 9 9 9

\hfill

rep(1:4, 2)

[1] 1 2 3 4 1 2 3 4

\hfill

rep(1:4, each = 2)

[1] 1 1 2 2 3 3 4 4

\hfill

rep(1:4, each = 2, times = 3)

[1] 1 1 2 2 3 3 4 4 1 1 2 2 3 3 4 4 1 1 2 2 3 3 4 4

\hfill

rep(1:4,1:4)

[1] 1 2 2 3 3 3 4 4 4 4




\end{frame}


\begin{frame}{C2}

$>>$ Generando secuencias de n\'umeros

\hfill

10:18

[1] 10 11 12 13 14 15 16 17 18

\hfill

18:10

[1] 18 17 16 15 14 13 12 11 10

\hfill

-0.5:8.5

[1] -0.5 0.5 1.5 2.5 3.5 4.5 5.5 6.5 7.5 8.5

\hfill

seq(0,1.5,0.2)

[1] 0.0 0.2 0.4 0.6 0.8 1.0 1.2 1.4

\hfill

seq(1.5,0,-0.2)

[1] 1.5 1.3 1.1 0.9 0.7 0.5 0.3 0.1

\hfill




\end{frame}

x.values $<-$ seq(min(x),max(x),(max(x)-min(x))/100)

x $<-$ rnorm(18,10,2)

seq(88,50,along=x)

\hfill

sequence(5)

[1] 1 2 3 4 5

\hfill

sequence(5:1)

[1] 1 2 3 4 5 1 2 3 4 1 2 3 1 2 1

\hfill

sequence(c(5,2,4))

[1] 1 2 3 4 5 1 2 1 2 3 4


\begin{frame}{C2}

$>>$ Sorting, Ranking and Ordering

\hfill

houses $<-$ read.table("c:\\temp \\houses.txt",header=T)

attach(houses)

names(houses)

[1] "Location" "Price"

\hfill

ranks $<-$ rank(Price)

sorted $<-$ sort(Price)

ordered $<-$ order(Price)

view $<-$ data.frame(Price,ranks,sorted,ordered)

view


\end{frame}


\begin{frame}{C2}

y

[1] 8 3 5 7 6 6 8 9 2 3 9 4 10 4 11

sample(y)

[1] 8 8 9 9 2 10 6 7 3 11 5 4 6 3 4

sample(y)

[1] 9 3 9 8 8 6 5 11 4 6 4 7 3 2 10

sample(y,5)

[1] 9 4 10 8 11

sample(y,5)

[1] 9 3 4 2 8

sample(y,replace=T)

[1] 9 6 11 2 9 4 6 8 8 4 4 4 3 9 3



\end{frame}


\begin{frame}{C2}

p <- c(1, 2, 3, 4, 5, 5, 4, 3, 2, 1)

x<-1:10

sapply(1:5,function(i) sample(x,4,prob=p))


\end{frame}


\begin{frame}{C2}

$>>$ Matrices

X $<-$ matrix(c(1,0,0,0,1,0,0,0,1),nrow=3)

X

class(X)

[1] "matrix"

attributes(X)

\$dim

[1] 3 3


vector $<-$ c(1,2,3,4,4,3,2,1)

V $<-$ matrix(vector,byrow=T,nrow=2)

V


dim(vector) $<-$ c(4,2)

is.matrix(vector)

[1] TRUE

\end{frame}



\begin{frame}{C2}

$>>$ Nombrando filas y columnas en matrices

\hfill

X $<-$ matrix(rpois(20,1.5),nrow=4)

X

\hfill

rownames(X) $<-$ rownames(X,do.NULL=FALSE,prefix="Trial.")

X

\hfill

drug.names<-c("aspirin", "paracetamol", "nurofen", "hedex", "placebo")

colnames(X) $<-$ drug.names

X

\hfill

dimnames(X) $<-$ list(NULL,paste("drug.",1:5,sep=""))

X




\end{frame}



\begin{frame}{C2}

\end{frame}



\begin{frame}{C2}

\end{frame}



\begin{frame}{C2}

\end{frame}



\begin{frame}{C2}

\end{frame}



\begin{frame}{C2}

\end{frame}



\begin{frame}{C2}

\end{frame}






\end{document}

